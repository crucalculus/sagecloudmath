\documentclass{article}
\usepackage{sagetex}

\begin{document}

Using Sage\TeX, one can use Sage to compute things and put them into
your \LaTeX{} document. For example, there are
$\sage{number_of_partitions(1269)}$ integer partitions of $1269$.
You don't need to compute the number yourself, or even cut and paste
it from somewhere.

I can also get the prime factorization of any number. $158760 = \sage{factor(158760)}$.

\begin{sageblock}
    g(x) = x^6-3*x^2+3*x-4;
    h(x) = (2*x - 4)^6;
\end{sageblock}

And simplify an algebraic expression like $\sage{g(x)} = \sage{factor(g(x))}$.

We can go the other way and take something like: \[(2x - 4)^6 = \sage{expand((2*x - 4)^6)}\]


Here's some Sage code:

\begin{sageblock}
    f(x) = exp(x) * sin(2*x)
\end{sageblock}

The second derivative of $f$ is

\[
  \frac{\mathrm{d}^{2}}{\mathrm{d}x^{2}} \sage{f(x)} =
  \sage{diff(f, x, 2)(x)}.
\]

Here's a plot of $f$ from $-1$ to $1$:

\sageplot{plot(f, -1, 1, figsize=4)}

\end{document}