\documentclass{article}
\usepackage[imagemagick]{sagetex}

\begin{document}

Using Sage\TeX, one can use Sage to compute things and put them into
your \LaTeX{} document. For example, there are
$\sage{number_of_partitions(1269)}$ integer partitions of $1269$.
You don't need to compute the number yourself, or even cut and paste
it from somewhere.

I can also get the prime factorization of any number. $158760 = \sage{factor(158760)}$.

Let's define two new functions in a sageblock:

\begin{sageblock}
    g(x) = x^6-3*x^2+3*x-4;
    h(x) = (2*x - 4)^6;
\end{sageblock}

And simplify an algebraic expression like $\sage{g(x)} = \sage{factor(g(x))}$.

We can go the other way and take something like: \[(2x - 4)^6 = \sage{expand((2*x - 4)^6)}\]


Here's some Sage code:

\begin{sageblock}
    f(x) = exp(x) * sin(2*x)
\end{sageblock}

The second derivative of $f$ is

\[
  \frac{\mathrm{d}^{2}}{\mathrm{d}x^{2}} \sage{f(x)} =
  \sage{diff(f, x, 2)(x)}.
\]

Here's a plot of $f$ from $-2$ to $2$:

\sageplot{plot(f, -2, 2, figsize=3)}

\newpage

% I don't know what I'm doing to include a 3d plot in a latex doc
% the online documentation suggests that the graphics engine can be
% troublesome and that Imagemagick is required to do the conversion
% Not sure I understand how this process works yet...

\begin{sagesilent}
    u,v = var("u,v")
    G = parametric_plot3d([cos(u)*v, sin(u)*v, 3/2-3*v/2], (u, 0, 2*pi), (v, 0, 1.5), opacity = 0.8, plot_points=[200,200]) # the cone
 # G = graphs.CubeGraph(5)
\end{sagesilent}

% need empty [] so sageplot knows you want png format, and aren't
% passing an option to includegraphics
\sageplot[][png]{G.plot3d(engine='tachyon')}

\end{document}
%sagemathcloud={"zoom_width":125}