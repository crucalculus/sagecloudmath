\documentclass{article}
\usepackage[imagemagick]{sagetex}
\usepackage{pgfplots}


\begin{document}

Partial fraction expansion demo.

\begin{sagesilent}
    var('x')
    m = (1/(x-2)) + (1/(x+2))
    simplified = m.full_simplify()
\end{sagesilent}

Start with $m(x) = \sage{m}$.

Ask SAGE to simplify it and display the result.

Simplified, $m(x) = \sage{simplified}$.

Using Sage\TeX, one can use Sage to compute things and put them into
your \LaTeX{} document. For example, there are\\
$\sage{number_of_partitions(1269)}$ integer partitions of $1269$.
You don't need to compute the number yourself, or even cut and paste
it from somewhere.

I can also get the prime factorization of any number. $158760 = \sage{factor(158760)}$.

Let's define two new functions in a sageblock:

\begin{sageblock}
    g(x) = x^6-3*x^2+3*x-4;
    h(x) = (2*x - 4)^6;
\end{sageblock}

And simplify an algebraic expression like$\sage{g(x)} = \sage{factor(g(x))}$.

We can go the other way and take something like: \[(2x - 4)^6 = \sage{expand((2*x - 4)^6)}\]


Here's some Sage code:

\begin{sageblock}
    f(x) = exp(x) * sin(2*x)
\end{sageblock}

The second derivative of $f$ is

\[
  \frac{\mathrm{d}^{2}}{\mathrm{d}x^{2}} \sage{f(x)} =
  \sage{diff(f, x, 2)(x)}.
\]

Here's a plot of $f$ from $-2$ to $2$:

\sageplot{plot(f, -2, 2, title=r'$f(x)$', figsize=3)}

\sageplot{polar_plot([2+3*sin(x)], (x, 0, 2*pi),color='black', figsize=[3,3])}

\newpage

% let's try and get a 3d plot by using techniques on high school math and chess
% I think the idea is just to use sage to calculuate a bunch of points, print them
% to an output string wrapped in a tikzpicture environment, then have tikz draw the graph.
\begin{sagesilent}
    x = var('x')
    y = var('y')
    step = 0.2
    x1 = -3
    x2 = 3
    y1 = -3
    y2 = 6.20
    
    output1 = ""
    output1 += r"\begin{tikzpicture}[scale=1.0]"
    output1 += r"\begin{axis}[xlabel=$x$,ylabel=$y$,zlabel=$z$,view={25}{30},xmin=%d, xmax=%d, ymin=%d, ymax=%d]"%(x1,x2,y1,y2-step)
    output1 += r"\addplot3[surf,mesh/rows=%d,opacity=0.75] coordinates {"%((y2-step-y1)/step+1)
    for y in srange(y1,y2,step):
        for x in srange(x1,x2,step):
            #output += ""
            output1 += r"(%f, %f, %f) "%(x,y,sin(x)*sin(y))
    
    output1 += r"};"
    output1 += r"\end{axis}"
    output1 += r"\end{tikzpicture}"
\end{sagesilent}

\sagestr{output1}

\end{document}
%sagemathcloud={"zoom_width":170}